% PS: \Chapter содержит \newpage. Учитывайте это.
\Chapter{Анализ в метрических пространствах}

\Paragraph{Метрические и векторные пространства}

\begin{defination}[Метрическое пространство]
	$X$ -- множество. Отображение $\rho: X \times X \to [0, +\infty)$
	\begin{enumerate}
		\item $\rho(x, y) = 0 \rla x = y$
		\item $\rho(x,y) = \rho(y,x)$
		\item $\rho(x,y) \le \rho(x,z) + \rho(z,y) \quad \forall x,y,z\in X$
	\end{enumerate}
	Тогда $(X,\rho)$ -- метрическое пространство.
\end{defination}

\begin{subpar}{Примеры метрик}
	\begin{enumerate}
		\item \textbf{Дискретная метрика (метрика лентяя).}
		
		\[ X\text{ -- произвольное множество, }\begin{cases}
			\rho(x,y) = 0,&x=y\\
			\rho(x,y)=1,&x\neq y
		\end{cases} \]
		
		\item \textbf{Обычная метрика.} $X = \RR,\ \rho(x,y)=\abs{x-y}$
	\end{enumerate}
\end{subpar}

\begin{defination}[Внутренность множества]
	\[ \Int A := \left\{ a \in A \Big| a \text{ -- внутрення точка множества } A \right\} \]
	
	Пример: $(\RR, \abs{x-y})$, $A = [0;1]$, $\Int A = (0,1)$; $B = \{0\} \cup \{7\} \cup [1;5)$, $\Int B = (1,5)$.
\end{defination}

\begin{theorem}[Свойства внутренности множества]
	\begin{th-base}
		\begin{enumerate}
			\item $\Int A \subset A$
			\item $\Int A = \bigcup\limits_{\alpha \in I} A_\alpha, \quad \text{где }A_\alpha$ -- открытое подмножество $A$.
			\item $\Int A$ -- открытое множество.
			\item $\Int A = A \rla A$ -- открыто.
		\end{enumerate}
	\end{th-base}
	\begin{th-proof}
		\begin{enumerate}
			\item Очевидно.
			\item \fbox{$\subset$} $a \in \Int A \ra \exists r > 0\ B_r(a) \subset A \ra B_r(a) \subset \bigcup\limits_{\alpha\in I} A_\alpha$\\
			\fbox{$\supset$} $a \in \bigcup\limits_{\alpha\in I} A_\alpha \ra \exists \beta:\ a \in A_\beta \ra \exists r\ B_r(a) \subset A_\beta \subset A \ra a \text{ -- внутренняя точка } A \ra a \in \Int A $
			\item Следствие второго свойства и свойства открытых множеств.
			\item \fbox{$\ra$} Так как $\Int A = A$ и $\Int A$ -- открыто, то и $A$ -- открыто.\\
				\fbox{$\la$} $A$ -- открыто, $\Int A = \bigcup\limits_{\alpha \in I} A_\alpha$. Так как $A \subset A$, то
				$\bigcup\limits_{\alpha \in I} A_\alpha \supset A$:
				\[ \begin{cases}
					\Int A \subset A\\
					\Int A \supset A
				\end{cases} \ra \Int A = A \] 
		\end{enumerate}
	\end{th-proof}
	% NOTE: когда-нибудь я сделаю подстветку, как и у формулировокок. Поэтому
	% можно будет потом писать это тупо через th-proof.
	% Эх.......
	\begin{th-so}[такие-то]
		Следствия...
	\end{th-so}
	\begin{th-example}[секие-то]
		Пример...
	\end{th-example}
\end{theorem}

% PS${{ random.randint(1, 5) }}: тут нужно кучу newlineов добавить ибо я додик и без этого ломается
% оглавнение. Между блоками \begin-\end тоже должны быть пустые строки.

